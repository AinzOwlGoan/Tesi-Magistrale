\section{Formule estese}
In riferimento al capitolo della cinematica e della dinamica andiamo ora a vedere le formule per esteso:
\subsection{Formule cinematica diretta accelerazione}
Andiamo a concentrarci sulle formule viste nel capitolo della cinematica diretta di accelerazione. [metti riferimento]
\begin{equation}
\dot{J}_{11} = A_{11} - B_{11} - C_{11} - D_{11}
\end{equation}
Andiamo ora a definire i parametri: 
\begin{equation*}
A_{11} = \frac{l(y-l\sin\theta_{2})(d\sin\theta_{1}-y\cos\theta _{1}+x\sin\theta _{1})\cdot (\dot{y}-l\dot{\theta_1}\cos\theta _{1}+A_{1.1.2})}{(d-x+l\cos\theta_{2})(y-l\sin\theta_{1}+\frac{(y-l\sin\theta _{2})(d+x-l\cos\theta _{1})}{d-x+l\cos\theta _{2}})^2}
\end{equation*}
\begin{equation*}
\begin{split}
A_{1.1.2} = \frac{(y-l\sin\theta_{2})(\dot{x}+l\dot{\theta_1}\sin\theta_{1})}{d-x+l\cos\theta_{2}}+\frac{\dot{y}-l\dot{\theta_2}\cos\theta_{2}(d+x-l\cos\theta_{1})}{d-x+l\cos\theta_{2}}+  \\
+ \frac{(y-l\sin\theta_{2})(\dot{x}+l\dot{\theta}_{2}\sin\theta_{2})(d+x-l\cos\theta_{1})}{{(d-x+l\cos\theta_{2})}^2}
\end{split}
\end{equation*}
\begin{equation*}
B_{11} =\frac{l(y-l\sin\theta_{2})(\dot{x}\sin\theta_{1}-\dot{y}\cos\theta_{1}+d\dot{\theta_1}\cos\theta_{1}+\dot{\theta_1}x\cos\theta_{1}+\dot{\theta_{1}}y\sin\theta _{1})}{(d-x+l\cos\theta_{2})\cdot(y-l\sin\theta_{1}+\frac{(y-l\sin\theta_{2})(d+x-l\cos\theta_{1})}{d-x+l\cos\theta_{2}})}
\end{equation*}
\begin{equation*}
C_{11} = \frac{l(\dot{y}-l\dot{\theta_{2}}\cos\theta_{2})(d\sin\theta_{1}
    -y\cos\theta_{1}+x\sin\theta_{1})}{(d-x+l\cos\theta_{2})(y-l\sin\theta_{1}+\frac{(y-l\sin\theta_{2})
    (d+x-l\cos\theta_{1})}{d-x+l\cos\theta_{2}})}
\end{equation*}
\begin{equation*}
D_{11} = \frac{l(y-l\sin\theta_{2})(\dot{x}+l\dot{\theta_2}\sin\theta_{2})
    (d\sin\theta_{1}-y\cos\theta_{1}+x\sin\theta_{1})}{(d-x+l\cos\theta_{2})^2(y-l\sin\theta_{1}+\frac{(y-l\sin\theta_{2})
    (d+x-l\cos\theta_{1})}{d-x+l\cos\theta _{2}})}
\end{equation*}
\begin{equation}
\dot{J}_{21} =\frac{l(d\sin\theta_{1}-y\cos\theta_{1}+x\sin\theta_{1})(\dot{y}-
 l\dot{\theta_1}\cos\theta_{1}+A_{1.1.2}}{sa}
\end{equation}
\subsection{Prerequisiti dinamica}\label{appendice:t34punto}
Le matrici importati sono $\dot{J}_3$ e $\dot{J}_4$ che andiamo a riprendere ora come:
\begin{equation}
\dot{J_3} =
\begin{bmatrix}
 	\dot{J}_{31} & \dot{J}_{32} \\ 
 	\dot{J}_{33} & \dot{J}_{34}
 \end{bmatrix}
\end{equation}
Con i relativi parametri pari a:
\begin{equation*}
\dot{J}_{31} = -l\cos\theta_1\dot{\theta_1}+\frac{1}{2}\cos\theta_3\dot{\theta_3}J_{34}+\frac{1}{2}\sin\theta_3\dot{J_{34}}
\end{equation*}
\begin{equation*}
\dot{J}_{32} =   -l\frac{1}{2}\cos\theta_3\dot{\theta_3}J34+\frac{1}{2}\sin\theta_3\dot{J_{34}} 
\end{equation*}
\begin{equation*}
\dot{J}_{33} =  -l\sin\theta_1\dot{\theta_1}-\frac{1}{2}\sin\theta_3\dot{\theta_3}J_{34}+\frac{1}{2}\cos\theta_3\dot{J_{34}}
\end{equation*}
\begin{equation*}
\dot{J}_{34} = -\frac{1}{2}l\sin\theta_3\dot{\theta_3}J_{34}+\frac{1}{2}\cos\theta_3\dot{J_{34}}
\end{equation*}
Invece, definiamo $\dot{J}_4$ in modo analogo come:
\begin{equation}
\begin{bmatrix}
\dot{J}_{41}  & \dot{J_{42}} \\ \dot{J_{43}} & \dot{J_{44}}
\end{bmatrix}
\end{equation}
Con parametri:
\begin{equation*}
\dot{J_{41}}= -l\frac{1}{2}\cos\theta_4\dot{\theta_4}J_{34}+\frac{1}{2}\sin\theta_4\dot{J_{34}}
\end{equation*}
\begin{equation*}
\dot{J_{42}}= -l\cos\theta_2\dot{\theta_2}+\frac{1}{2}\cos\theta_4\dot{\theta_4}J_{34}+\frac{1}{2}\sin\theta_4\dot{J_{34}}
\end{equation*}
\begin{equation*}
\dot{J_{43}}=  -l\frac{1}{2}\sin\theta_4\dot{\theta_4}J_{34}+\frac{1}{2}\cos\theta_4\dot{J_{34}}          
\end{equation*}
\begin{equation*}
\dot{J_{44}}= -l\sin\theta_2\dot{\theta_2}-\frac{1}{2}\sin\theta_4\dot{\theta_4}J_{34}+\frac{1}{2}\cos\theta_4\dot{J_{34}}
\end{equation*}