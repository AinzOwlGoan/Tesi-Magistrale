\section{Conclusioni}
\subsection{Confronto modellazione teorica-pratica}
L'obiettivo principale della tesi è stato lo studio e l'analisi di approcci di controllo centralizzato al fine di trovare quello migliore da implementare nel manipolatore.
\\Dopo la trattazione sulla cinematica, dinamica e cinetostatica del sistema è stato possibile definire le relazioni matematiche che caratterizzano le due parti del manipolatore, braccia e vite, il workspace ed i possibili punti raggiungibili ed implementare quanto visto su un modello simulato. Il modello una volta validato grazie a processi di cosimulazione (Matlab e Adams) ha fornito una base per l'implementazione degli schemi di controllo.
\par L'implementazione di differenti tecniche ha permesso di ottenere una maggior conoscenza nell'ambito della manipolazione del 5R e la fase di simulazione ha permesso di definire quale tra le strategie analizzate fosse la più consona, andando a scartare l'impiego di altri criteri; uno in particolare è stato il controllo robusto, influenzato da fenomeni vibratori generati dalla commutazione ad alta frequenza della variabile di controllo.
\par Per concludere, la validazione sperimentale ha permesso di evidenziare che il criterio ottimo di controllo è quello a dinamica inversa, in quanto con questa tipologia si riesce ad ottenere un \textit{trade-off} tra prestazioni e  robustezza. L'ultima fase è stata quella di test, andando ad implementare traiettorie bidimensionali e tridimensionali ed analizzare come il controllo andava a comportarsi.
\subsection{Sviluppi futuri}
Gli obiettivi futuri legati al progetto di tesi riguardano nuovi criteri di controllo, andando ad analizzare tipologie di controllo diverse che potrebbero essere implementate ed dare risultati migliori di quella dinamica inversa. Un'altra tematica riguarda il telecontrollo del manipolatore, che assumerà il ruolo di slave, da un'interfaccia aptica con sensore di forza.

