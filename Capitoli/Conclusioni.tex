\section{Conclusioni}
L'obiettivo principale della tesi è stato lo studio e l'analisi di approcci di controllo centralizzato al fine di trovare quello migliore da implementare nel manipolatore.
\\Dopo la trattazione sulla cinematica, dinamica e cinetostatica del sistema è stato possibile definire le relazioni matematiche che caratterizzano le due parti del manipolatore, braccia e vite, il workspace ed i possibili punti raggiungibili implementando quanto visto su un modello simulato. Il modello una volta validato grazie a processi di cosimulazione (Matlab e Adams) ha fornito una base per l'implementazione degli schemi di controllo.
\par L'implementazione di differenti tecniche ha permesso di ottenere una maggior conoscenza nell'ambito della manipolazione del 5R e la fase di simulazione ha permesso di definire quale tra le strategie analizzate fosse la più consona, andando a scartare l'impiego di altri criteri; uno in particolare è stato il controllo robusto, influenzato da fenomeni vibratori generati dalla commutazione ad alta frequenza della variabile di controllo.
\par Per concludere, la validazione sperimentale ha permesso di evidenziare che il miglior criterio di controllo è quello a dinamica inversa, in quanto si riesce ad avere un \textit{trade-off} tra prestazioni e  robustezza. L'ultima fase è stata quella di test, implementando traiettorie bidimensionali e tridimensionali ed analizzando il comportamento del controllo.
\subsection*{Sviluppi futuri}
\addcontentsline{toc}{subsection}{Sviluppi futuri}
Gli obiettivi futuri legati al progetto di tesi riguardano nuovi criteri di controllo, andando ad analizzare tipologie di controllo diverse che potrebbero essere implementate ed dare risultati migliori di quello a dinamica inversa. Un'altra tematica riguarda il telecontrollo del manipolatore mediante interfaccia aptica (che assumerà il ruolo di master).

