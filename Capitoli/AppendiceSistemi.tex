\section{Cenni sistemi dinamici}
In questa sezione dell'appendice  verrà considerato un sistema non lineare come quello del manipolatore e verrà analizzata la stabilità dell'equilibrio a tempo discreto.
\\Dato un sistema dinamico, a dimensione finita, stazionario, di tipologia MIMO (\textit{multiple input multiple output}), non lineare e a tempo discreto è possibile descriverlo mediante l'equazione:
\begin{equation}
\xi(k+1) = f(\xi(k),u(k))
\end{equation}
Il sistema può andare ad evolversi in due stati:
\begin{itemize}
\item Movimento d'equilibrio, chiamato anche movimento nominale: $\tilde{\xi}(k) = \overline{\xi}(k)$, si ottiene applicando l'ingresso nominale di equilibrio $\tilde{u}(k) = \overline{u}(k)$ al sistema posto nello stato iniziale $\tilde{\xi}(k_0 = 0)=\overline{\xi}$ in questo modo il movimento soddisfa
\begin{equation}
\tilde{\xi}(k+1) = \overline{\xi} = f(\overline{\xi},\overline{u})
\end{equation}
\item Ingresso perturbato $\xi(k)$ ottenuto applicando la stessa tipologia di ingresso nominale però in uno stato differente $\xi_0 \neq \overline{\xi}$, allora il movimento soddisfa il seguente sistema di equazioni:
\begin{equation}
\begin{cases}
\xi(k+1) = f(\xi(k),u(k)) \\
\xi(k_0=0) = \xi_0
\end{cases}
\end{equation}
\end{itemize}
Viene definita la perturbazione sullo stato del sistema come la differenza fra i due movimenti in particolare è possibile esprimerla come:
\begin{equation}
\delta \xi(k) = \xi(k)-\overline{\xi} \in \mathbb{R}^n 
\end{equation}
procedendo in maniera incrementale si trova:
\begin{equation}
\delta \xi(k+1) = \xi(k+1)+\overline{\xi} = f(\xi(k),\overline{u})-\overline{\xi} = f(\overline{\xi}+\delta\xi(k),\overline{u})-\overline{\xi}
\end{equation}
risulta essere non lineare, si può trovare la condizione iniziale come:
\begin{equation*} 
\delta \xi(k_0 = 0) = \xi_0 -\overline{\xi} \rightarrow \delta \xi \neq 0
\end{equation*}
\\La soluzione all'equazione non è facile da trovare in quanto dipende da un punto preciso chiamato punto di equilibrio, identificato da $(\overline{\xi},\overline{u})$; in caso di sistemi non lineari stazionari, la proprietà di stabilità riguarda esclusivamente un intorno dello stato di equilibrio considerato e non l'intero sistema.
\subsection{Stabilità asintotica}\label{Appendice:stabilita}
A partire dallo stato di equilibrio $\xi$ del sistema dinamico a tempo discreto visto precedentemente, la condizione sufficiente perché risulti asintoticamente stabile è che 
\begin{equation}
\forall i : |\lambda_i(A)| < 1
\end{equation}
Con $A = \frac{\partial f(\xi,\overline{u})}{\partial \xi}\bigg|_{\xi = \overline{\xi}}$
\\In questo caso esiste un intorno dell'equilibrio ($\Psi_{\overline{\xi}}$) tale per cui, data qualunque perturbazione iniziale $\delta \xi (k) \in \Psi_{\overline{\xi}}$ la perturbazione sullo stato $\delta \xi (k)$ rimane limitata nel tempo ed asintoticamente tende a zero. 
\begin{equation}
\delta \xi (k+1) = f(\overline{\xi}+\delta \xi (k),\overline{u}) = A\delta\xi(k)+h(\delta\xi(k))
\end{equation}
Da tutto questo è possibile derivare anche la condizione di instabilità, esprimibile come:
\begin{equation}
\exists i : |\lambda_i(A)|>1
\end{equation}
con A definita come prima. In questo caso, non esiste alcun introno dell'equilibrio che fa rimanere la perturbazione dello stato limitata a partire da una perturbazione iniziale $\delta \xi_0 \in \Psi_{\overline{\xi}}$. 
\subsection{Metodo delle differenze finite}\label{appendix:diff}
In caso di equazioni differenziali ordinarie l'interesse è quello di risolvere in un preciso dominio, in particolare, per poterle risolvere numericamente bisogna imporre la condizione che il dominio sia limitato. Le differenze finite vengono utilizzate per risolvere equazioni differenziali che descrivono fenomeni stazionari, ovvero quando le quantità di riferimento variano nello spazio (ipotizzando che non varino nel tempo). La trattazione seguente utilizzerà come esempio l'equazione di Poisson
\begin{equation}
- \frac{d^2 u(\gamma)}{d\gamma^2}  =f(\gamma)
\end{equation}
con $\gamma\in[0,R]$ e $f:\mathbb{R}\rightarrow \mathbb{R}$ una funzione con condizioni $u(0)=a, u(R)=b$; per poterla risolvere bisogna garantire un minimo di regolarità per la funzione f. La prima idea applicata è la discretizzazione del dominio, ovvero la divisione dell'intervallo $[0,R]$ in intervalli sufficientemente piccoli la cui unione restituisce il dominio iniziale. Si definisce quindi la discretizzazione come:
\begin{equation}
\Gamma = \{\gamma_1 = 0 < \gamma_2 < \dots < \gamma_{N-1} < \gamma_N = R\}
\end{equation}
tale che $\cup_{i=1}^{N-1} [\gamma_i,\gamma_{i+1}] = [0,R]$. Essendo la discretizzazione costruita con distanze uguali si definisce $\Delta \gamma_k = \gamma_{k+1} -\gamma_k$. Riprendendo il concetto di rapporto incrementale per la definizione della derivata prima, sapendo che la derivata di una funzione è definita come il limite del rapporto incrementale in caso che sia finito, ne consegue che è possibile definire la derivata seconda come:
\begin{equation*}
u''(\gamma) = \lim_{\Delta\gamma \to 0} \frac{u'(\gamma+\Delta\gamma) -u'(\gamma)}{\Delta \gamma}
\end{equation*}
Introducendo quindi $u(\gamma_k) \approx u_k$, cioè, quello che viene calcolato è l'approssimazione della soluzione calcolata in $\gamma_k$. Per fornire un'approssimazione alle differenze finite di queste derivate si procede con l'idea di trovare un intervallo di discretizzazione piccolo tale per cui prendendo due elementi $\gamma_k$ e $\gamma_{k+1}$ distano sufficientemente poco. È possibile approssimare la derivata come:
\begin{equation}
u'(\gamma_k) \approx \frac{u_{k+1}+u_k}{\Delta \gamma}
\end{equation}
L'approssimazione ottenuta però non è molto raffinata, si può ottenere un'espressione migliore, del secondo ordine effettuando una differenza finita centrata ottenendo quindi:
\begin{equation}
u'(\gamma_k) \approx \frac{u_{k+1}-u_{k-1}}{2\Delta \gamma}
\end{equation}